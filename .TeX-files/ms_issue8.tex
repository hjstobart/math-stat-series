\documentclass[11pt]{article}
\usepackage[margin=1.2in]{geometry} 
\usepackage{amsmath}
\usepackage{tcolorbox}
\usepackage{amssymb}
\usepackage{amsthm}
\usepackage{lastpage}
\usepackage{fancyhdr}
\usepackage{accents}
\usepackage{parskip}
\pagestyle{fancy}
\setlength{\headheight}{40pt}

\begin{document}

\lhead{Mr. \textsc{H. Stobart}} 
\rhead{\textsc{Math/Stat Series \\ Issue 8, Aug-22}}
\cfoot{\thepage\ of \pageref{LastPage}}

\begin{tcolorbox}
\begin{center}
    \large
    \textsc{Partial Differential Equations \\ Solving the 1D Wave Equation}
\end{center}
\end{tcolorbox}

\begin{center}
\textbf{Note:} \textit{This work is intended for informative and educational purposes only.}
\end{center}

\section*{1. Introduction}
In last month's issue we solved the one-dimensional heat/diffusion equation––one of the most frequently examined partial differential equations. Continuing in that spirit I now want to solve what is perhaps the second most frequently considered PDE: the one-dimensional wave equation.

\section*{2. The Setup}
The general form of the wave equation is given by the following PDE.
\begin{equation}
    \frac{\partial^2 u}{\partial t^2} = c^2 \frac{\partial^2 u}{\partial x^2}.
\end{equation}
Where $c$ is some constant. 

The first thing you will notice about the wave equation is that we now have the second partial derivative with respect to time $t$, whereas we only had a first order derivative with the heat/diffusion equation. Beyond that the equations look very similar, although we have redefined the diffusivity constant ($D$) to $c^2$ now. 

\section*{3. General Solution}
At the highest level we find that the equation,
\begin{equation}
    u(x,t) = f(x-ct) + g(x+ct)
\end{equation}
is the most general solution. We can see this is the case by taking the respective derivatives
\begin{align}
    \frac{\partial^2 u}{\partial t^2} &= c^2 f''(x-ct) + c^2 g''(x+ct) \\
    \frac{\partial^2 u}{\partial x^2} &= f''(x-ct) + g''(x+ct).
\end{align}
Thus the result follows. 

But how do we go about solving the PDE when we are given some \textbf{Initial Conditions}? Suppose we have the following,
\begin{align}
    u(x,0) &= \phi(x) \\
    \frac{\partial}{\partial t} u(x,0) &= \psi(x).
\end{align}

Then using our general solution, given by (2) we find that,
\begin{align}
    f(x) + g(x) &= \phi(x) \\
    -c f'(x) + c g'(x) &= \psi(x). 
\end{align}

Let's now take (8) and integrate,
\begin{align}
    -c f'(x) + c g'(x) &= \psi(x) \\[10pt]
    f'(x) - g'(x) &= -\frac{1}{c} \psi(x) \\[10pt]
    f(x) - g(x) &= -\frac{1}{c} \int_{a}^{x} \psi(\xi) d\xi.
\end{align}
Where we have specifically chosen $a$ such that the constants of integration do not introduce any additional terms (for simplicity).

We now have two equations in two unknowns and, hence, we can solve for both $f(x)$ and $g(x)$ respectively. Rearranging (11) to make $g(x)$ the subject, we obtain
\begin{equation}
    g(x) = f(x) +\frac{1}{c} \int_{a}^{x} \psi(\xi) d\xi.
\end{equation}

Now substituting (12) into (7) to eliminate $g(x)$ we find that,
\begin{align}
    f(x) + f(x) +\frac{1}{c} \int_{a}^{x} \psi(\xi) d\xi  &= \phi(x) \\[10pt]
    2 f(x) &= \phi(x) -\frac{1}{c} \int_{a}^{x} \psi(\xi) d\xi \\[10pt]
    f(x) &= \frac{1}{2} \left[ \phi(x) -\frac{1}{c} \int_{a}^{x} \psi(\xi) d\xi \right].
\end{align}

By a similar argument we find that,
\begin{equation}
    g(x) = \frac{1}{2} \left[ \phi(x) +\frac{1}{c} \int_{a}^{x} \psi(\xi) d\xi \right].    
\end{equation}

\newpage

Now returning to our general solution (2). We can write the full solution to the wave equation,
\begin{align}
    u(x,t) &= \frac{1}{2} \left[ \phi(x-ct) + \phi(x+ct) \right] + \frac{1}{2c} \left[ \int_{a}^{x+ct} \psi(\xi) d\xi + \int_{x-ct}^{a} \psi(\xi) d\xi \right] \\[10pt]
    u(x,t) &= \frac{1}{2} \left[ \phi(x-ct) + \phi(x+ct) \right] + \frac{1}{2c} \left[ \int_{x-ct}^{x+ct} \psi(\xi) d\xi \right]
\end{align}

This is called \textbf{d'Alembert's Solution}. Notice that we have not included any boundary conditions, this means the solution is valid on the entire real line, $\mathbb{R}$. 

\section*{4. Discussion}
As we have seen the solution to the wave equation appears much nicer than that of the heat/diffusion equation. Of course, we have only considered the general case for both, if we had been provided with specific functions for either the initial or boundary conditions (or both), then our analysis and subsequent solution would have required a great deal more work. It's not necessarily extremely difficult, although that largely depends on the functions given, and if you work through the steps in a sensible way then you should be fine. 

\end{document}
